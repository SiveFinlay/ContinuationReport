\documentclass[12pt,a4paper]{article}
\usepackage{enumerate} %put in numbers or bullet points
\usepackage{setspace}
%\raggedright %justify the text on the left only
% NC: I removed this so you get indented paragraphs which makes it easier to read
\usepackage{graphicx} 	% For adding pictures
\usepackage{float}
\usepackage{pdflscape}	% for landscape pages
\pagenumbering{arabic}	% Page numbers
\usepackage{fancyhdr} % add headers and footers
\usepackage{hyperref} % hyper links for references

\onehalfspacing %1.5 line spacing
\usepackage[round]{natbib} % author-year citations in round brackets

%End of preamble
%--------------------------------------------------

\begin{document}

\title{Patterns of morphological evolution in tenrecs}
\author{}
\date{}
\maketitle


%Add a header 
\renewcommand{\headrulewidth}{0.0pt}
\thispagestyle{fancy}				%header on the first page only
\lhead{Sive Finlay, Progress Report}
\chead{}
\rhead{September 2014}


\section{Overview}
% NC: Haven't changed much here, just corrected typos and gave some general comments. Be careful with typos. There were quite a lot and they are all highlighted by the spell check.


% NC: I wonder if this would be better framed the other way around - so start with the fact that you don't enjoy research and don't want to pursue it as a career, thus don't see any benefit to forcing yourself to complete something that would likely cause a lot of stress. Then move on to the technical difficulties? Written this way round it seems like you got overwhelmed, rather than having a better series of reasons. Also this probably doesn't need to be quite so long. You can discuss this in the meeting, and both parties are aware of your reasons. They will probably want you to re-discuss in the meeting anyway.

	% NC: \\ and then a line space forces a line break between paragraphs if you think it looks neater.
	% NC: `xxx' will give you nice quotation marks
	% NC: \noindent will remove indents if you prefer
	% NC: I don't mind, I'm just showing you some stuff that you can do, it may help with the thesis
	

	My research is an investigation of evolutionary patterns in tenrecs (Afrosoricida,  Tenrecidae). After much careful consideration and seeking advice from many different sources, I have decided to convert my project to a research masters thesis rather than PhD. My plan is to use the attached paper (disparity within tenrecs) as the basis for the thesis along with a complementary analysis of convergence among tenrecs and other small mammals. 
	
	I have not made the decision to change direction lightly but I believe that it is the best option for me. My decision to switch to a masters does not come from being overwhelmed or unwilling to approach hard work. Despite the volume of work remaining I know that I do have the ability to complete the project if necessary. However, my strengths and interests lie outside traditional research and I don't wish to continue struggling to finish a piece of work to which I'm not suited.
	
	I am extremely grateful to the committee for all of your help and guidance over the past two years. I feel especially thankful and fortunate to have had such a dedicated supervisor: I am so grateful to Natalie for her constant mentoring, teaching and advice and especially for her support as I was making this recent decision.  % aww 
	
	If I were to continue with my project then officially I would have one year left to complete the PhD. However, realistically I know that it would take much longer. My plan was to have seven chapters: four of which would be paper driven along with introductory, data collection and conclusions chapters. The paper which accompanies this report would be one of the four paper-driven chapters. The others would be 1) a review of methods of quantifying convergence, 
	2) quantifying morphological convergence among tenrecs and other small mammals using multiple methods and 3) testing whether morphological convergence is predicted by ecological similarity.
    In my first year report in November 2013 I included an outline for a further chapter based on behavioural studies of echolocatory capabilities in shrew-type (\textit{Microgale}) tenrecs. Natalie and I travelled to Madagascar in April of this year but unfortunately our experiments were unsuccessful so I was unable to collect sufficient data for this chapter.
    %I took out the reference to attaching my first year report, I think I'm sending them enough documents already.
	
	Completing the PhD thesis plan would be more than a simple extension of the work that I'm doing already. I'm including an analysis of convergence in my MSc thesis but this is nowhere near as extensive as the analyses which would be required for a PhD. In addition, the review paper of methods of measuring convergence would require a level of understanding and familiarity with the different statistical methods which I know I would struggle to acquire. Furthermore, to complete the PhD I would need to collect an entirely new data set of ecological variables for my species and then learn and apply different statistical methods for assessing the evidence for ecological convergences among tenrecs.

	I would be very grateful for advice about what's the best way to approach the transition from a PhD to MSc. My current plan is to contact the IRC this month (September) to tell them that I would like to terminate my contract and that they should close my research account with College. My fees are covered by my final year of `Schols' so I would like to re-register as a student for this academic year and submit before the second annual deadline set by the graduate studies office on 1st of March. However, instead of allowing the write up to drag on for the best part of a year, Natalie and I have set our own deadline of submitting the thesis before the end of January. I have attached a separate document to this report which describe my thesis structure and plans for completion of each chapter.
 
	Here I outline the current progress on my paper and a summary of my other on-going work. The separate document is an annotated table of contents for my thesis. It includes the structure of each chapter, a description of the amount of work remaining, estimated lengths for each section and the expected timeline for completion.  


%---------------------------------------------------------------
\section{Progress on the accompanying paper}

	\textit{Journal targets: Journal of Evolutionary Biology, PLoS ONE, Journal of Mammalian Evolution?} 
\bigskip

	Over the summer, and while I was deciding between the MSc vs. PhD option, I have been working on my analyses of disparity within tenrecs in the format of a paper. I have completed the majority of the analyses and I have attached the draft paper to this report. All that remains from the analytical side is to add a short sensitivity analysis for morphometric error checking.  However, I know that the both the introduction and discussion need to be developed more before they will be ready for submission.
	
	Now that I have decided to take the MSc route, I think it would be more productive and efficient to write a more extensive introduction and discussion for the thesis and then distil these down to relevant sections for a paper rather than the other way around. Therefore I plan to finish and submit my thesis before returning to getting the paper ready for submission next year. 

	My aim for the paper is that it is a test of a broad principle; the importance of testing our assumptions about phenotypic variation in groups that are considered to be exceptionally diverse, rather than a more limited study of shape variation in one group of mammals. Given that tenrecs are often cited as an example of a phenotypically diverse group, my finding that they are not significantly more diverse than their closest relatives was unexpected and therefore should make an interesting paper. I tried to put an adaptive radiation slant on my findings about tenrec morphological diversity. However, I don't have any explicit measure of the `adaptiveness' of cranial shape so I have also tried to be careful not to suggest that my analyses should be treated as a test of whether tenrecs are an adaptive radiation or not.
	
	I would welcome any comments or suggestions you might have for how I could make the overall story clearer and more interesting to a wide audience.
	

	
% NC: apostrophes get clumsy in these cases so I'd go with tenrec morphological convergence (rather than tenrecs' for simplicity of reading)


%--------------------------------------------------------------- %I could put this back in but I didn't think it was relevant 
% NC: Up to you, might be nice to include :)
\section{Other work}
	
	\begin{enumerate}

	\item \textbf{Publication since the November 2013 report}\\
		Healy, K., Guillerme T., \textbf{Finlay, S.,}, Kane, A., Kelly, S.B.A., McClean, D., Kelly, D.J., Donohue, I., Jackson, A.L. and \\Cooper, N., (2014).\\
		Ecology and mode-of-life explain lifespan variation in birds and mammals. \textit{Proceedings of the Royal Society B, 281(1784)} 

	\item \textbf{Presentations}\\
		This summer I gave my first oral conference presentations at the Evolution conference in Raleigh, North Carolina (June) and the British Ecological Society Macroecology conference in Nottingham (July). I will also present my research at the Midland's Science Festival in November. 

	\item \textbf{Reviewing}\\
		So far I have reviewed a journal article (International Journal of Primatology) and a book chapter on phylogenetic comparative methods for quantifying convergence.
		
	\item \textbf{Teaching}\\	
		I have a great opportunity to exapand my teaching skills by workings as the Biology tutor for the Trinity Access Programme. I will be teaching four hours of lectures per week for the first semester and possibly continuing into the second semester depending on how long this maternity cover position lasts. I will also give one lecture on convergent evolution to the Conservation and Biodiversity MSc students in October.

	\end{enumerate}
%----------------------------------------------------------------

%GANNT chart : I've moved this to the table of contents file



\end{document}