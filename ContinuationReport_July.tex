\documentclass[12pt,a4paper]{article}
\usepackage{enumerate} %put in numbers or bullet points
\usepackage{setspace}
\raggedright %justify the text on the left only
\usepackage{graphicx} 	% For adding pictures
\usepackage{float}
\usepackage{pdflscape}	% for landscape pages
\pagenumbering{arabic}	% Page numbers
\usepackage{fancyhdr} % add headers and footers
\usepackage{hyperref} % hyper links for references

\onehalfspacing %1.5 line spacing
\usepackage[round]{natbib} % author-year citations in round brackets

%End of preamble
%--------------------------------------------------

\begin{document}

\title{Patterns of morphological evolution in tenrecs}
\author{}
\date{}
\maketitle


%Add a header 
\renewcommand{\headrulewidth}{0.0pt}
\thispagestyle{fancy}				%header on the first page only
\lhead{Sive Finlay, Progress Report}
\chead{}
\rhead{September 2014}


\section{Overview}

	My research is an investigation of evolutionary patterns in tenrecs (Afrosoricida,  Tenrecidae). The original aim of my project was to quantify both morphological and ecological diversity within tenrecs (disparity) and their similarities to other small insectivorous mammals (convergence). I have almost completed the first part of this plan; please see the paper draft "Qunatifying cranial morphological disparity in tenrecs (Afrosoricida, Tenrecidae) with implications for their designation as an adaptive radiation" which accompanies this report.

	After much careful consideration and seeking advice from many different sources, I have decided to convert my project to a Masters by research thesis rather than PhD. My plan is to use the attached paper as the basis for the thesis along with a complementary analysis of convergence among tenrecs and other small mammals. 

	I have not made the decision to change direction lightly but I believe that it is the best option for me. If I were to continue with my project then officially I would have one year left to complete the PhD. However, realistically I know that it would take much longer. My plan was to have seven chapters: four of which would be paper driven along with introductory, data collection and conclusions chapters. The accompanying paper would be one of the four paper-driven chapters. The others would be 1) a review of methods of quantifying convergence, 2) quantifying morphological convergence among tenrecs and other small mammals using multiple methods, 3) testing whether morphological convergence is predicted by ecological similarity. In my first year report in November 2013 (see attached document) I included an outline for a further chapter based on behavioural studies of echolocatory capabilities in shrew-type (\textit{Microgale}) tenrecs. Natalie and I travelled to Madagascar in April of this year but unfortunately our experiments were unsuccessful so I was unable to collect sufficient data for this chapter.

	Completing this PhD thesis plan would be more than a simple extension of the work that I'm doing already. I would need to master different statistical techniques for quantifying convergence to a sufficient degree to be able to both use the methods for my own research and to critique them in a review. Although I'm planning to include an analysis of convergence in my masters thesis this will not be as extensive. Furthermore, to comlete the PhD I would need to collect an entirely new data set of ecological variables for my species and then learn and apply different statistical methods for assessing the evidence for ecological convergences among tenrecs.

	I understand that it's common to feel overwhelmed by a PhD half way through the project. However, my decision to switch to a masters instead is not purely a matter of being overwhelmed or unwilling to approach hard work. Despite the volume of work remaining I know that I do have the ability to complete the project if necessary. However, my strengths and interests lie outside traditional research and I don't wish to continue struggling to finish a piece of work to which I'm not suited.

	I am extremely grateful to the committee for all of your help and guidance over the past two years. I feel especially thankful and fortunate to have had such a dedicated supervisor: I am so grateful to Natalie for her constant mentoring, teaching and advice and especially for her support as I was making this recent decision.   

	I would be very grateful for advice about what's the best way to approach this transition. My current plan is to contact the IRC this month (September) to tell them that I would like to terminate my contract and that they should close my research account with college.  Then I will work on finishing my paper on tenrec disparity and complete the separate analysis of convergence among tenrecs and other small mammals. Adding an introduction and discussion should not take too much longer because I have already done the majority of the background research that's necessary for these sections.

	My fees are covered by my final year of Schols so I would like to re-register as a student for this academic year and submit before the second annual deadline set by the graduate studies office on 1st of March. However, instead of allowing the write up to drag on for the best part of a year, Natalie and I have set our own deadline of submitting the thesis before the end of January (see the Figure \ref{gantt} GANTT chart at the end of this report for a my planned timeline for comletion).
 
	In this report I outline the current progress on my paper, plans for how I will develop the work into a complete Master's thesis and a brief summary of the other academic work completed since my last report.  

%Add in the table of contents from my current thesis file?
%---------------------------------------------------------------
\section{Progress on the accompanying paper}

	\textit{Journal targets: Journal of Evolutionary Biology, PLoS ONE, Journal of Mammalian Evolution? , submit November 2014.} 

	I have completed the majority of the work for this paper, all that remains is to develop the introduction and discussion. \\
	Given that tenrecs are often cited as an example of a phenotypically diverse group \citep[e.g.][]{Olson2013}, my finding that they are not significantly more diverse than their closest relatives was unexpected and therefore should make an interesting paper.

	I would very much welcome any comments on the draft paper, particularly if you have suggestions for how I could make the overall story clearer and more interesting to a wide audience. My aim for the paper is that it's a test of a broad principle; the importance of testing our assumptions about phenotypic variation in groups that are considered to be exceptionally diverse, rather than a more limited study of shape variation in one group of mammals. I tried to put an adaptive radiation slant on my findings about tenrecs' morphological diversity. However, I don't have any explicit measure of the "adaptiveness" of cranial shape so I have been careful not to suggest that my analyses should be treated as a test of whether tenrecs are an adaptive radiation or not.
%------------------------------------------------------------------



\section{Completing the rest of the thesis}

	I have attached a draft copy of the early stages of my thesis to this report. I'm not asking the committee to read through the whole document but the table of contents do show the outline and structure of the thesis. I have already filled in some of the sections and here I will outline my plans for completing the rest of the thesis.

\subsection{Introduction}

	I will frame the thesis as a general study of patterns of phenotypic diversity using tenrecs as an example.
	I will discuss disparity within the framework of how it relates to the study of adaptive radiations \citep{Losos2010a} and convergence within the context of what it tells us about the repeatability of evolution \citep[e.g][]{Blount2008}. In particular, I will stress the importance of taking quantiative approaches to studying each of these patterns rather than relying on subjective estimates. I will follow these discussions with a brief introduction of tenrecs and the long-standing interest in the apparently high morphological and ecological diversity within the family and the similarities among tenrecs and other distantly related mammal species \citep[e.g.][]{Eisenberg1969, Soarimalala2011, Olson2013}. 

	I have already completed most of the background research for this section so it will not take long to put the information together. Part of it will also be an expansion of the introduction that I have written for my disparity paper. I will add the sections on convergence and more general interests of phenotypic diversity in October and the introduction will be completed by the end of November (figure \ref*{gantt}).

\subsection{Methods}

	This is a large section but I have completed both the work and writing for most of it already.
	I have divided the methods up into four parts: 1) museum data collection, 2) geometric morphometric analyses, 3) error checking and 4) data analysis for disparity and convergence studies (see the table of contents in the accompanying thesis draft).

	I have completed all of sections 1 and 2 with the exception of creating diagrams to demonstrate the linear measurements which I took for skulls and limbs. For the error checking part I still need to run senstitivity analyses to test the accuracy of landmark placement on my pictures. I have all the data prepared for these analyses so they will not take long to complete. I will finish the error checking and add the results to the supplementary section of my disparity paper by the end of this month (figure \ref{gantt}).
	I have completed the analysis and write up for measuring morphological disparity in tenrecs (see accompanying paper).	
	Initially I tried to quantify convergence among tenrecs and other small mammals in January/February of this year but I moved onto the disparity study instead because I found some of the convergence methods difficult to implement. However, by working on the disparity analyses for the past few months my coding skills are now far better than before so I think that I will find the convergence analyses both easier and quicker when I return to them. I'm also no longer trying to compare multiple measures of convergence so this more focused approach will make the analyses faster as well. I complete these analyses in September and October and finish writing up the methods and results by the end of November (figure \ref{gantt}).
	
	%Put in details about how I will quantify convergence  
 
\subsection{Results}
	I will divide the results into two sections: disparity (results section from my paper) and convergence. I have already added my disparity results to the thesis draft.
	The output of my convergence analyses will be principal component plots showing the morphospace occupied by each of the families; tenrecs, golden moles, hedgehogs, shrews, solenodons and moles.
	
	%Put in more details once I've figured out the analyses, maybe example graphs too?

\subsection{Discussion}

	Here I will summarise my findings and interpret the results within the context of what they tell us about the diversity within tenrecs and the similarities among tenrecs and other small mammal species. I will also highlight the importance of taking a quantitative approach to studies of evolutionary diversity among species groups. In particular I will stress the need to apply existing methods to new groups of species which are usually not as well studied as the groups that are used to develop such metrics (JUSTIFICATION REFS). 
	My suggestions for future directions will be based on the plans that I had made for continuing with the PhD. This will include the need to apply multiple metrics for measuring convergence (REFS), analyses of ecological similarities to test whether there are correlations between ecological and phenotypic convergence (REFS) and the need to measure
	functional importance and "adaptiveness" of convergent traits \citep{Losos2010}. I will also include a brief mention of my unsuccessful fieldwork experiments for testing echolocatory capabilities in \textit{Microgale} tenrecs and the potential for future studies of behavioural convergences among tenrecs and other small mammals.
	
	Some of this discussion will be based my disparity paper and the future directions part will be an expansion of the plans that I had already outlined in previous reports. I will complete the rest of this section in November/December so that I will have a full thesis draft by the end of December which will leave me ready to submit by the end of January (figure \ref{gantt}).


%----------------------------------------------------------------
%Echolocation: I don't think this is relevant any more but I've kept it as a basis for the future directions part of my thesis
%\section{Unsuccessful fieldwork for testing behavioural convergence}

	%In my November 2013 report I outlined my plans for incorporating tests of behavioural convergence into my project. I was interested in whether early reports of an ability to echolocate in some tenrecs \citep{Gould1965} may be extended to other species of the \textit{Microgale} (shrew-type) tenrecs and therefore provide evidence of behavioural convergence between tenrecs and shrews \citep{Siemers2009}. Natalie and I went to Madagascar in March/April as part of a research trip led by Dr.Steve Goodman to conduct behavioural tests of echolocation in \textit{Microgale}. Our aim was to record the sounds made by the animals as they moved through a wooden maze towards a food reward to determine whether there was evidence that they were using sounds to navigate through their environment. We tried multiple variations of our protocol but unfortunately none of the animals we tested produced any noise (17 individuals from 5 different species). It is clear that this negative result is a failure of our experiment rather than an indication that \textit{Microgale} don't navigate using sounds. Our sample included \textit{Microgale dobsoni} which are one of the few species which are known to echolocate from previous experiments \citep{Gould1965}. Similarly, other more experienced researchers in the group had heard the \textit{Microgale} making sounds while foraging. Previous studies of echolocation in small mammals \citep{Gould1964, Gould1965, Tomasi1979, Siemers2009} all used captive individuals which were trained to perform specific tasks. Such a prolonged procedure was not possible wihtin our constraints of time and facilities.

	%Given that I have no useable data to address my questions of behavioural convergence it is not enough information for a complete chapter. I could mention the experiments briefly in my concluding chapter as an idea for further directions. Alternatively, I could leave these investigations of behavioural convergence out of the thesis altogether to avoid detracting from my more general studies of morphological and ecological (dis)similarities among species.
%-----------------------------------------------------------------------

%Could probably take this out too
\section{Other work}
	
	\begin{enumerate}

	\item \textbf{Publications since the November 2013 report}\\
		\textbf{Finlay, S.}, Goodman, S.M., Cooper, N. Significant levels of morphological disparity in the adaptive radiation of tenrecs (Afrosoricida, Tenrecidae). \textit{In prep. To be submitted to the Journal of Evolutionary Biology}\\
	
	\bigskip 
	
		Healy, K., Guillerme T., \textbf{Finlay, S.,}, Kane, A., Kelly, S.B.A., McClean, D., Kelly, D.J., Donohue, I., Jackson, A.L. and Cooper, N., 2014.Ecology and mode-of-life explain lifespan variation in birds and mammals. \textit{Proceedings of the Royal Society B, 281(1784)} 

	\item \textbf{Presentations}\\
		This summer I gave my first oral conference presenations at the Evolution conference in Raleigh, North Carolina (June) and the British Ecological Society Macroecology conference in Nottingham (July). 

	\item \textbf{Reviewing}\\

		So far I have reviewed a journal article (International Journal of Primatology) and a book chapter on phylogenetic comparative methods for quantifying convergence. I hope that I will get some more reviewing practice once I start publishing my own papers.

	\end{enumerate}
%----------------------------------------------------------------

%GANNT chart
\begin{landscape}
  \begin{figure}[p]
	\centering
	\includegraphics[keepaspectratio=true]{Gannt_July.png}
	\caption{Timeline for completion and submission of my thesis. Tasks are colour coded according to the accompanying key}
	\label{gantt}
  \end{figure}
\end{landscape}
%-------------------------------------------------------


\bibliographystyle{jeb}
\bibliography{refs_thesis} 


\end{document}