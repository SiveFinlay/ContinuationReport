\documentclass[12pt,a4paper]{article}
\usepackage[round]{natbib} % author-year citations in round brackets

%Practice change for git

\usepackage{hyperref}

\raggedright %justify the text on the left only
\usepackage{enumerate} %put in numbers or bullet points
\usepackage{setspace}
\onehalfspacing %1.5 line spacing
\begin{document}

\title{Steering committee progress report\\
Provisional thesis title; "Convergence and disparity in the evolution of tenrecs"}
\author{Sive Finlay}
\maketitle

\section{Introduction}
My PhD is an investigation of evolutionary patterns in tenrecs (Afrosoricida, Tenrecidae). My aim is to quantify both morphological and ecological diversity within the tenrec family (disparity) and their similarities to other insectivore mammals (convergence). 
I am now half way through my three year PhD and I have a plan of how my thesis will take shape. I included a provisional thesis outline of 5 chapters in my first year report in November 2013 (see attached document). However, based on the committee's advice that some of the chapters were too large and the fact that my fieldwork experiments did not go according to plan (see section 3 below), I have now modified the thesis plan.  The new plan is to have 7 chapters in my thesis. At least 4 of these chapters will be paper-driven. I have included a draft outline of my first paper, "Cranial morphological disparity within the adaptive radiation of tenrecs (Afrosoricida, Tenrecidae) is no greater than expected by chance" which will comprise my 3rd chapter.
Here I will outline my current progress and plans for completion of each chapter. 

\section{Thesis outline}
\begin{enumerate}
\item \textbf{Chapter 1: Introduction}\\

This is a general introductory chapter. I will include brief outlines of convergence and disparity and why they are useful and interesting measures of evolutionary diversity.
Along with the conclusions (chapter 8), this will be the last chapter that I write because it will be easier to introduce the rest of the thesis once I know how those chapters have taken shape. 

\item \textbf{Chapter 2: Data collection}
My thesis will be based on two main data sources; morphological and ecological. The questions addressed in chapters 4, 6 and 7 will all use these same data sources. Therefore, rather than repeating the information in multiple sections, I will combine my data collection methods into this single chapter which I can then refer to in the rest of the thesis. 
I have collected all of the data and completed most of the writing for the morphological section of this chapter. I haven't started the sections relating to the ecological data but I will work on this aspect as I collect the relevant data over the coming months (see Chapter 7 below).


%Don't do a separate chapter on methods of measuring disparity; put that information into the start of chapter 3


\item \textbf{Chapter 3: Quantifying morphological disparity in tenrecs}\\
\textit{Target: Journal of Evolutionary Biology or PLoS ONE}\\
This is the chapter which corresponds to the draft paper that accompanies this report. I have completed the majority of the work for this paper. Given that tenrecs are often cited as an example of a phenotypically diverse group \citep[e.g.][]{Olson2013}, my finding that morphological disparity is no greater than expected by chance was unexpected. Therefore I'm currently checking the code that I used for the calculations and simulation studies to ensure that the results are accurate rather than just artefacts of a fault in the methods. 

%Remember that this meeting will be at the end of May so hopefully I should have these coding checks done by then

I would very much welcome any comments on the draft paper, particularly if you have suggestions for how I could make the overall story clearer and more interesting to a wide audience. My aim for the paper is that it's a test of a broad principle; the importance of testing our assumptions about phenotypic variation in adaptively radiated groups, using a specific example of tenrecs rather than a more limited study of morphological variation in a particular group of mammals.

Dr. Steve Goodman, an expert in tenrec ecology and evolution, has expressed an interest in collaborating on the paper so I will send it to him for comments before submitting to the Journal of Evolutionary Biology.

\item \textbf{Chapter 4: Review of methods of detecting convergence}\\
\textit{Target: Journal of Evolutionary Biology}

Convergence is a common evolutionary pattern of historical significance and interest in evolutionary biology because it relates to questions of the relative roles of determinism and contingency in evolution \citep[c.f][]{Blount2008}. Many studies of convergence are purely qualitative descriptions of apparent patterns of similarity among species \citep[e.g.][]{Ben-Moshe2001, Leal2002, Fleischer2008} including recent studies of genomic convergence \citep{Jones2012, Parker2013}. 

However, there is increasing interest in developing new metrics of quantifying the degree of convergent evolution among species. These quantitative approaches for assessing the significance of apparent convergence are particularly important in the wake of research which revealed that some level of convergence is expected to evolve by chance in most phylogenies \citep{Stayton2008}.
The discrepancies among these metrics have produced a confusing variety of techniques of quantifying convergence with no clear guidelines for the relative suitabilities of different methods or how they compare to each other.
   

This chapter will be a review of exising methods for quantifying convergence. It will serve as both a background for my subsequent chapter on quantifying convergence in tenrecs and it will also be a separate paper. 
I will start with a brief review of why convergence is important and interesting to study and the developing need for rigorous quantitative approaches. I will classify convergence studies into three primary groups; papers which are purely qualitative/descriptive (see examples above), papers which quantify convergence and papers which fall between the two extremes - they don't just look at patterns of convergence but they don't measure significance either  \citep[e.g.][]{Tseng2013, Kawahara2013, Wroe2007, Jones2007, Clark2005}. I will group the papers that do quantify convergence into methods for measuring convergence across a tree \citep{Stayton2008}, between species pairs \citep{Muschick2012, Stayton2006, Harmon2005} or lineages of species \citep{Revell2007} and methods for quantifying community/faunal convergence in \citep{Burd2014, Alvarado2013, Ingram2013, Mahler2013, Moen2013, Segar2013, Elias2008, Melville2006}.  
While many of these metrics are seemingly sound and useful, they are usually developed on relatively 


Most of the methods were developed using simulation studies and tested on relatively few biological groups, for which detailed morphological, ecological and phylogenetic data are well known. The realistic potential for application of each method to other data sets and questions is often more restricted than the authors suggest. Therefore I will summarise the approach of each method,comment on any potential problems and outline their suitability for other data sets and questions. This review will be useful for the rest of my own research but it will make a topical and interesting paper.  

There are many existing papers which review conceptual insights and investigations of convergent evolution \citep[e.g.][]{Losos2011, Conway-Morris2006, Scheffer2006} but my paper will be the first review of methods of measuring convergence. 

I have an outline plan of the chapter and I have already completed an initial literature review. I will add to this research when I'm using some of these methods to quantify convergence in my own data (see chapter 5) and I will write the paper after I have completed my own analyses because that will give me a better understanding of the methods.

%This is a week end to the section, I need to come back and fix it 


\item \textbf{Chapter 5: Quantifying morphological convergence among tenrecs and other "insectivore" mammals}
\textit{Target; Journal of Evolutionary Biology}

Following on from my review of methods of quantifying convergence (chapter 4) I will apply some of these methods to my data to measure the degree of morphological similarity among tenrecs and other "insectivore" mammals.
As I described previously in my November 2013 report, I have an extensive morphological data base of tenrecs and the mammals they resemble. I have completed my geometric morphometric analyses of all of the skull and mandible specimens and I have moved onto quantifying convergence.


Put in which methods I'm going to use - both for overall convergence of the family and between specific groups.

I will write these analyses into a paper about the general importance of quantifying convergence focusing on the specific example of tenrecs. Despite numerous claims and indications that tenrecs are convergent other animal groups \citep[e.g.][]{Olson2013, Eisenberg1969, Soarimalala2011} this paper will be the first to test the hypothesis in a quantitative framework.

I think that my cranial analyses will be sufficient data for the paper. However, I also have a dataset of limb measurements which a final-year undergraduate is going to analyse for evidence of convergence this year. The results of her project will indicate whether it would be productive for me to include a straightforward analysis of my limb measurements (for example correlations between limb proportions and locomotion style/habitat) in my assessment of convergence.


\item \textbf{Chapter 6: Is morphological convergence between tenrecs and other "insectivore" mammals predicted by ecological similarity?}



\item \textbf{Chapter 7: Conclusions}
This chapter will summarise the importance of taking a quantitative approach to studies of evolutionary diversity (morphological and ecological) among species groups. In particular I will highlight the need to apply existing methods to new groups of species which are not usually not as well studied as groups which are used to develop such metrics. I will also include suggestions for future directions, particularly in the area of addressing the functional importance rather than pure description of convergent traits (put in the Losos reference here). I could also include mention my unsuccessful fieldwork experiments here (see below) as another potential basis for future work. 

\end{enumerate}

\section{Unsuccessful fieldwork for testing behavioural convergence}
In my November 2013 report I outlined my plans for incorporating tests of behavioural convergence into my project. I was interested in whether early reports of an ability to echolocate in some tenrecs \citep{Gould1965} may be extended to other species of the \textit{Microgale} (shrew-type) tenrecs and therefore provide evidence of behavioural convergence between tenrecs and shrews \citep{Siemers2009}. Natalie and I went to Madagascar in March/April as part of a research trip led by Dr.Steve Goodman to conduct behavioural tests of echolocation in \textit{Microgale}. Our aim was to record the sounds made by the animals as they moved through a wooden maze towards a food reward to determine whether there was evidence that they were using sounds to navigate through their environment. We tried multiple variations of our protocol but unfortunately none of the animals we tested produced any noise (17 individuals from 5 different species). It is clear that this negative result is a failure of our experiment rather than an indication that \textit{Microgale} don't navigate using sounds. Our sample included \textit{Microgale dobsoni} which are one of the few species which are known to echolocate from previous experiments \citep{Gould1965}. Similarly, other more experienced researchers in the group had heard the \textit{Microgale} making sounds while foraging. Previous studies of echolocation in small mammals \citep{Gould1964, Gould1965, Tomasi1979, Siemers2009} all used captive individuals which were trained to perform specific tasks. Such a prolonged procedure was not possible wihtin our constraints of time and facilities.

Given that I have no useable data to address my questions of behavioural convergence it is not enough information for a complete chapter. I could mention the experiments briefly in my concluding chapter as an idea for further directions. Alternatively, I could leave these investigations of behavioural convergence out of the thesis altogether to avoid detracting from my more general studies of morphological and ecological (dis)similarities among species.

\section{Other work}
\begin{enumerate}

\item \textbf{Publications since the November 2013 report}\\
\textbf{Finlay, S.}, Goodman, S.M., Cooper, N. Significant levels of morphological disparity in the adaptive radiation of tenrecs (Afrosoricida, Tenrecidae). \textit{In prep. To be submitted to the Journal of Evolutionary Biology}\\
\bigskip %skip a line between the two references
Healy, K., Guillerme T., \textbf{Finlay, S.,}, Kane, A., Kelly, S.B.A., McClean, D., Kelly, D.J., Donohue, I., Jackson, A.L. and Cooper, N., 2014.Ecology and mode-of-life explain lifespan variation in birds and mammals. \textit{Proceedings of the Royal Society B, 281(1784)} 

\item \textbf{Presentations}\\
This summer I will give my first oral conference presenations at the Evolution conference in Raleigh, North Carolina (June) and the British Ecological Society Macroecology conference in Nottingham (July).

\item \textbf{Reviewing}\\

So far I have reviewed a journal article (International Journal of Primatology) and a book chapter on phylogenetic comparative methods for quantifying convergence. I hope that I will get some more reviewing practice once I start publishing my own papers.

\end{enumerate}

\section{Timeline for PhD completion}
Put in the GANNT chart here

\bibliographystyle{jeb}
\bibliography{Refs_01_05_14_edited} %This is my edited JabRef file of all my references exported from EndNote. I had to edit the formatting for the special characters, capital letters and also change the file encoding.


\end{document}