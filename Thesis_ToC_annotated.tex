\documentclass[12pt,a4paper]{article}
\usepackage{enumerate} %put in numbers or bullet points
\usepackage{setspace}
%\raggedright %justify the text on the left only
\usepackage{graphicx} 	% For adding pictures
\usepackage{float}
\usepackage{pdflscape}	% for landscape pages
\pagenumbering{arabic}	% Page numbers
\usepackage{fancyhdr} % add headers and footers
\usepackage{hyperref} % hyper links for references

\onehalfspacing %1.5 line spacing
\usepackage[round]{natbib} % author-year citations in round brackets

%End of preamble
%--------------------------------------------------
%Annotated thesis table of contents to accompany the continuation meeting report in September 2014
\begin{document}

\title{MSc Thesis Outline}
\author{}
\date{}
\maketitle



\section{Introduction}
	There will be five subsections in the introduction. I have completed drafts of sections \ref{int_disp}, \ref{int_tenrecs} and \ref*{int_struc}. I estimate that the full chapter will be around five pages long and I will finish a full draft before (DATE).

	\subsection{Patterns of morphological diversity}
		I will introduce my research within the context of other studies which are interested in understanding patterns of phenotypic variation, particularly morphological differences measured with a geometric morphometrics approach. 

	\subsection{Disparity}
		\label{int_disp}
		Leading on from my introduction about morphological diversity, I will discuss how groups which show high levels of disparity (diversity of form) are particularly interesting when it comes to understanding the factors that lead to adaptive radiations. This section will be partly based on the introduction in my paper draft.  
		
	\subsection{Convergence}
		While disparity assesses the diversity of morphological form, morphological convergence takes the opposite approach by measuring the phenotypic similarities among distantly related species. I will include a brief summary of why convergent evolution continues to attract such great interest among researchers and the particular importance of taking quantitative approaches towards measuring degrees of convergence.
	
	\subsection{Tenrecs}
		\label{int_tenrecs}
		The introductory sections above will lead me in to introducing tenrecs as an example of a group which appears to show both morphological diversity and examples of convergence among other small mammal species. In particular, I will show that tenrecs are often used as an example of both an adaptively radiated group which also appears to be convergent with other species and yet these assumptions about their morphological diversity have not been tested previously. 
			 
	\subsection{Structure and contents of the thesis}
		\label{int_struc} 
		Finally, I will outline the remaining sections of my thesis: data collection, separate chapters for my analyses of disparity and convergence and then a general discussion chapter with future directions at the end.


\section{Data collection and processing}
	I'm using the same morphological data and general geometric morphometric analyses for both chapters \ref{sect_disp} and \ref*{sect_conv}. Therefore, I will combine all of this information into this one data chapter to avoid later repetition. I have completed most of this section but there are a few additional diagrams and analyses to add. I expect the completed chapter to be approximately 25 pages (due to the number of figures and tables, not just text!) and I will finish the first draft by (DATE).

	\subsection{Data collection}
		This section includes details of the species I measured, data collected, linear measurements and methods for photographing and processing images. I have finished all of this section with the exception of creating diagrams to depict the linear measurements for skulls and limbs.

	\subsection{Geometric morphometric analyses}
		Here I describe all of my morphometric analyses including landmark choice, placement and General Procrustes superimposition analyses. I have completed the analyses, diagrams and write up for all of this section.
		
	\subsection{Error checking}
		Finally, I will discuss the measures I included in my methods to deal with potential taxonomic, identification, measurement and morphometrics errors. I have completed half of this section but I still need to run a sensitivity analysis to check the accuracy of landmark placement on my pictures. I have all of the data prepared so they will not take long to complete.

\section{Disparity in tenrecs compared to their closest relatives}
	\label{sect_disp}

\section{Convergence among tenrecs and other small mammals}
	\label{sect_conv}

\section{Discussion}




\end{document}